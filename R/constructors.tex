\PassOptionsToPackage{unicode=true}{hyperref} % options for packages loaded elsewhere
\PassOptionsToPackage{hyphens}{url}
%
\documentclass[]{article}
\usepackage{lmodern}
\usepackage{amssymb,amsmath}
\usepackage{ifxetex,ifluatex}
\usepackage{fixltx2e} % provides \textsubscript
\ifnum 0\ifxetex 1\fi\ifluatex 1\fi=0 % if pdftex
  \usepackage[T1]{fontenc}
  \usepackage[utf8]{inputenc}
  \usepackage{textcomp} % provides euro and other symbols
\else % if luatex or xelatex
  \usepackage{unicode-math}
  \defaultfontfeatures{Ligatures=TeX,Scale=MatchLowercase}
\fi
% use upquote if available, for straight quotes in verbatim environments
\IfFileExists{upquote.sty}{\usepackage{upquote}}{}
% use microtype if available
\IfFileExists{microtype.sty}{%
\usepackage[]{microtype}
\UseMicrotypeSet[protrusion]{basicmath} % disable protrusion for tt fonts
}{}
\IfFileExists{parskip.sty}{%
\usepackage{parskip}
}{% else
\setlength{\parindent}{0pt}
\setlength{\parskip}{6pt plus 2pt minus 1pt}
}
\usepackage{hyperref}
\hypersetup{
            pdftitle={constructors.R},
            pdfauthor={sarah},
            pdfborder={0 0 0},
            breaklinks=true}
\urlstyle{same}  % don't use monospace font for urls
\usepackage[margin=1in]{geometry}
\usepackage{color}
\usepackage{fancyvrb}
\newcommand{\VerbBar}{|}
\newcommand{\VERB}{\Verb[commandchars=\\\{\}]}
\DefineVerbatimEnvironment{Highlighting}{Verbatim}{commandchars=\\\{\}}
% Add ',fontsize=\small' for more characters per line
\usepackage{framed}
\definecolor{shadecolor}{RGB}{248,248,248}
\newenvironment{Shaded}{\begin{snugshade}}{\end{snugshade}}
\newcommand{\AlertTok}[1]{\textcolor[rgb]{0.94,0.16,0.16}{#1}}
\newcommand{\AnnotationTok}[1]{\textcolor[rgb]{0.56,0.35,0.01}{\textbf{\textit{#1}}}}
\newcommand{\AttributeTok}[1]{\textcolor[rgb]{0.77,0.63,0.00}{#1}}
\newcommand{\BaseNTok}[1]{\textcolor[rgb]{0.00,0.00,0.81}{#1}}
\newcommand{\BuiltInTok}[1]{#1}
\newcommand{\CharTok}[1]{\textcolor[rgb]{0.31,0.60,0.02}{#1}}
\newcommand{\CommentTok}[1]{\textcolor[rgb]{0.56,0.35,0.01}{\textit{#1}}}
\newcommand{\CommentVarTok}[1]{\textcolor[rgb]{0.56,0.35,0.01}{\textbf{\textit{#1}}}}
\newcommand{\ConstantTok}[1]{\textcolor[rgb]{0.00,0.00,0.00}{#1}}
\newcommand{\ControlFlowTok}[1]{\textcolor[rgb]{0.13,0.29,0.53}{\textbf{#1}}}
\newcommand{\DataTypeTok}[1]{\textcolor[rgb]{0.13,0.29,0.53}{#1}}
\newcommand{\DecValTok}[1]{\textcolor[rgb]{0.00,0.00,0.81}{#1}}
\newcommand{\DocumentationTok}[1]{\textcolor[rgb]{0.56,0.35,0.01}{\textbf{\textit{#1}}}}
\newcommand{\ErrorTok}[1]{\textcolor[rgb]{0.64,0.00,0.00}{\textbf{#1}}}
\newcommand{\ExtensionTok}[1]{#1}
\newcommand{\FloatTok}[1]{\textcolor[rgb]{0.00,0.00,0.81}{#1}}
\newcommand{\FunctionTok}[1]{\textcolor[rgb]{0.00,0.00,0.00}{#1}}
\newcommand{\ImportTok}[1]{#1}
\newcommand{\InformationTok}[1]{\textcolor[rgb]{0.56,0.35,0.01}{\textbf{\textit{#1}}}}
\newcommand{\KeywordTok}[1]{\textcolor[rgb]{0.13,0.29,0.53}{\textbf{#1}}}
\newcommand{\NormalTok}[1]{#1}
\newcommand{\OperatorTok}[1]{\textcolor[rgb]{0.81,0.36,0.00}{\textbf{#1}}}
\newcommand{\OtherTok}[1]{\textcolor[rgb]{0.56,0.35,0.01}{#1}}
\newcommand{\PreprocessorTok}[1]{\textcolor[rgb]{0.56,0.35,0.01}{\textit{#1}}}
\newcommand{\RegionMarkerTok}[1]{#1}
\newcommand{\SpecialCharTok}[1]{\textcolor[rgb]{0.00,0.00,0.00}{#1}}
\newcommand{\SpecialStringTok}[1]{\textcolor[rgb]{0.31,0.60,0.02}{#1}}
\newcommand{\StringTok}[1]{\textcolor[rgb]{0.31,0.60,0.02}{#1}}
\newcommand{\VariableTok}[1]{\textcolor[rgb]{0.00,0.00,0.00}{#1}}
\newcommand{\VerbatimStringTok}[1]{\textcolor[rgb]{0.31,0.60,0.02}{#1}}
\newcommand{\WarningTok}[1]{\textcolor[rgb]{0.56,0.35,0.01}{\textbf{\textit{#1}}}}
\usepackage{graphicx,grffile}
\makeatletter
\def\maxwidth{\ifdim\Gin@nat@width>\linewidth\linewidth\else\Gin@nat@width\fi}
\def\maxheight{\ifdim\Gin@nat@height>\textheight\textheight\else\Gin@nat@height\fi}
\makeatother
% Scale images if necessary, so that they will not overflow the page
% margins by default, and it is still possible to overwrite the defaults
% using explicit options in \includegraphics[width, height, ...]{}
\setkeys{Gin}{width=\maxwidth,height=\maxheight,keepaspectratio}
\setlength{\emergencystretch}{3em}  % prevent overfull lines
\providecommand{\tightlist}{%
  \setlength{\itemsep}{0pt}\setlength{\parskip}{0pt}}
\setcounter{secnumdepth}{0}
% Redefines (sub)paragraphs to behave more like sections
\ifx\paragraph\undefined\else
\let\oldparagraph\paragraph
\renewcommand{\paragraph}[1]{\oldparagraph{#1}\mbox{}}
\fi
\ifx\subparagraph\undefined\else
\let\oldsubparagraph\subparagraph
\renewcommand{\subparagraph}[1]{\oldsubparagraph{#1}\mbox{}}
\fi

% set default figure placement to htbp
\makeatletter
\def\fps@figure{htbp}
\makeatother


\title{constructors.R}
\author{sarah}
\date{2020-09-26}

\begin{document}
\maketitle

Creates an initial data.table of adult agents. Since this is an initial
spread, some data is made up/dummy: for example, we don't know the ID of
the parents. Latitude/Longitude of agents is randomly determined through
another function init\_position and added to the data.table after
construction.

@section Data.table variables and initialisation: \itemize{
\item{ID:     Unique ID number of agent.}} gender: Male is 0, female is
1. Sampled by 1 Binomial trial as opposed to a Bernoulli trial as
Bernoulli requires another package. mateID: Unique ID of their mate.
Since no initial mosquitoes will have a mate yet, it is initialised as
-1. Males will always have mateID as -1 since they can have multiple
mates. enzyme: Enzyme Kinetic Score. See init for explanation.
Initialised uniform randomly. age: Age in days. Initialised uniform
randomly around what we would expect young adults to be. gonoCycle:
Gonotrophic cycle. Means something different for males and females.
males: number of times they've mated in a day, to be reset daily
females: how many times they've laid a clutch of eggs we roughly
estimate the gonoCycle of females based on age. timeDeath: Timestep they
died: initialised as -1 as they are alive. typeDeath: Random
mortality/trapped death/death due to old age: which type? whereTrapped:
in the event of trapped death, where did they die? -1 otherwise.
motherID: Unique ID of mother. -1 for initial adult data.table fatherID:
Unique ID of father. -1 for initial adult data.table. infStatus: 1 if
they carry Wolbachia, 0 if no Wolbachia, -1 for CI for initial wild type
we assume they all start with 0. lat: Initial north/south or `y'
coordinate of agent. Should start with -16. long: Initial east/west of
`x' coordinate of agent. Should start with 145. @param N The number of
initial adult agents. @param pmale Probability of being male. @return A
data.table of N adult agents.

\begin{Shaded}
\begin{Highlighting}[]
\NormalTok{initialise_adults <-}\StringTok{ }\ControlFlowTok{function}\NormalTok{(N,pmale)\{}
\NormalTok{  noVariables <-}\StringTok{ }\DecValTok{13}
\NormalTok{  init.dt <-}\StringTok{ }\KeywordTok{setNames}\NormalTok{(}\KeywordTok{data.frame}\NormalTok{(}\KeywordTok{matrix}\NormalTok{(}\DataTypeTok{ncol =}\NormalTok{ noVariables, }\DataTypeTok{nrow =}\NormalTok{ N)),}
                 \KeywordTok{c}\NormalTok{(}\StringTok{"ID"}\NormalTok{, }\StringTok{"gender"}\NormalTok{,}\StringTok{"mateID"}\NormalTok{, }\StringTok{"enzyme"}\NormalTok{,}\StringTok{"age"}\NormalTok{,}\StringTok{"gonoCycle"}\NormalTok{,}\StringTok{"timeDeath"}\NormalTok{,}\StringTok{"typeDeath"}\NormalTok{,}\StringTok{"whereTrapped"}\NormalTok{,}\StringTok{"motherID"}\NormalTok{,}\StringTok{"fatherID"}\NormalTok{,}\StringTok{"infStatus"}\NormalTok{,}\StringTok{"releaseLoc"}\NormalTok{))}

\NormalTok{  init.dt}\OperatorTok{$}\NormalTok{gender       <-}\StringTok{ }\KeywordTok{as.integer}\NormalTok{(}\KeywordTok{lapply}\NormalTok{(init.dt}\OperatorTok{$}\NormalTok{gender, }\ControlFlowTok{function}\NormalTok{(x) x<-}\StringTok{ }\KeywordTok{rbinom}\NormalTok{(}\DecValTok{1}\NormalTok{,}\DecValTok{1}\NormalTok{,}\DecValTok{1}\OperatorTok{-}\NormalTok{pmale))) }\CommentTok{#Need to convert to not be a list}
\NormalTok{  init.dt}\OperatorTok{$}\NormalTok{ID           <-}\StringTok{ }\DecValTok{1}\OperatorTok{:}\NormalTok{N}
\NormalTok{  init.dt}\OperatorTok{$}\NormalTok{age          <-}\StringTok{ }\KeywordTok{as.integer}\NormalTok{(}\KeywordTok{lapply}\NormalTok{(init.dt}\OperatorTok{$}\NormalTok{age, }\ControlFlowTok{function}\NormalTok{ (x) x<-}\KeywordTok{round}\NormalTok{(truncnorm}\OperatorTok{::}\KeywordTok{rtruncnorm}\NormalTok{(}\DecValTok{1}\NormalTok{,}\DataTypeTok{mean=}\DecValTok{20}\NormalTok{,}\DataTypeTok{sd=}\DecValTok{2}\NormalTok{,}\DataTypeTok{a=}\DecValTok{14}\NormalTok{,}\DataTypeTok{b=}\DecValTok{30}\NormalTok{))))}
\NormalTok{  init.dt}\OperatorTok{$}\NormalTok{motherID     <-}\StringTok{ }\DecValTok{-1}
\NormalTok{  init.dt}\OperatorTok{$}\NormalTok{fatherID     <-}\StringTok{ }\DecValTok{-1}
\NormalTok{  init.dt}\OperatorTok{$}\NormalTok{releaseLoc   <-}\StringTok{ }\DecValTok{-1}
\NormalTok{  init.dt}\OperatorTok{$}\NormalTok{timeDeath    <-}\StringTok{ }\DecValTok{-1}
\NormalTok{  init.dt}\OperatorTok{$}\NormalTok{typeDeath    <-}\StringTok{ }\DecValTok{-1}
\NormalTok{  init.dt}\OperatorTok{$}\NormalTok{infStatus    <-}\StringTok{ }\DecValTok{0}
\NormalTok{  init.dt}\OperatorTok{$}\NormalTok{whereTrapped <-}\StringTok{ }\DecValTok{-1}
\NormalTok{  init.dt}\OperatorTok{$}\NormalTok{mateID       <-}\StringTok{ }\DecValTok{-1}
\NormalTok{  init.dt}\OperatorTok{$}\NormalTok{enzyme       <-}\StringTok{ }\KeywordTok{as.double}\NormalTok{(}\KeywordTok{lapply}\NormalTok{(init.dt}\OperatorTok{$}\NormalTok{enzyme, }\ControlFlowTok{function}\NormalTok{(x) x <-}\StringTok{ }\KeywordTok{runif}\NormalTok{(}\DecValTok{1}\NormalTok{, }\DataTypeTok{min=}\DecValTok{0}\NormalTok{, }\DataTypeTok{max=}\DecValTok{1}\NormalTok{)))  }\CommentTok{#CHANGE: want to use initial age distribution to get spread of initial enzymes}

  \CommentTok{#FIX this shit and make it work better :(}
\NormalTok{  init.dt}\OperatorTok{$}\NormalTok{gonoCycle[}\KeywordTok{which}\NormalTok{(init.dt}\OperatorTok{$}\NormalTok{gender }\OperatorTok{==}\StringTok{ }\DecValTok{0}\NormalTok{)] <-}\StringTok{ }\DecValTok{0} \CommentTok{#males start off at 0 because 'gonoCycle' tracks number of mating events in a day for males}
\NormalTok{  init.dt}\OperatorTok{$}\NormalTok{gonoCycle[}\KeywordTok{which}\NormalTok{(init.dt}\OperatorTok{$}\NormalTok{gender }\OperatorTok{==}\StringTok{ }\DecValTok{1} \OperatorTok{&}\StringTok{ }\NormalTok{init.dt}\OperatorTok{$}\NormalTok{age }\OperatorTok{>=}\DecValTok{14} \OperatorTok{&}\StringTok{ }\NormalTok{init.dt}\OperatorTok{$}\NormalTok{age }\OperatorTok{<}\StringTok{ }\DecValTok{20}\NormalTok{)] <-}\StringTok{ }\DecValTok{0}
\NormalTok{  init.dt}\OperatorTok{$}\NormalTok{gonoCycle[}\KeywordTok{which}\NormalTok{(init.dt}\OperatorTok{$}\NormalTok{gender }\OperatorTok{==}\StringTok{ }\DecValTok{1} \OperatorTok{&}\StringTok{ }\NormalTok{init.dt}\OperatorTok{$}\NormalTok{age }\OperatorTok{>=}\DecValTok{20} \OperatorTok{&}\StringTok{ }\NormalTok{init.dt}\OperatorTok{$}\NormalTok{age }\OperatorTok{<}\StringTok{ }\DecValTok{26}\NormalTok{)] <-}\StringTok{ }\DecValTok{1}
\NormalTok{  init.dt}\OperatorTok{$}\NormalTok{gonoCycle[}\KeywordTok{which}\NormalTok{(init.dt}\OperatorTok{$}\NormalTok{gender }\OperatorTok{==}\StringTok{ }\DecValTok{1} \OperatorTok{&}\StringTok{ }\NormalTok{init.dt}\OperatorTok{$}\NormalTok{age }\OperatorTok{>=}\DecValTok{26}\NormalTok{)] <-}\StringTok{ }\DecValTok{2}

  \CommentTok{#initial position of each mosquito, then combine with rest of dataframe}
\NormalTok{  position.dt <-}\StringTok{ }\KeywordTok{init_position}\NormalTok{(boundaryDat,N)}
\NormalTok{  init.dt     <-}\StringTok{ }\KeywordTok{cbind}\NormalTok{(init.dt,position.dt)}

  \KeywordTok{return}\NormalTok{(init.dt)}
\NormalTok{\}}
\end{Highlighting}
\end{Shaded}

Creates an initial data.table of juvenile agents. Since this is an
initial spread, some data is made up/dummy: for example, we don't know
the ID of the parents. Latitude/Longitude of agents is randomly
determined and added to the data.table after construction.

@section Data.table variables and initialisation: \describe{
\item{motherID:}{ID of mother.}
\item{fatherID:}{ID of father.}
\item{age:}{Age in days. Initialised uniform randomly around what we would expect
             young adults to be.}
\item{stage:}{Development stage of clutch. 1: egg, 2: larvae, 3: pupae.}
\item{infProb:}{Probability of carrying Wolbachia. 0: no Wolbachia, -1: Cytoplasmic
              Incompatability, else infProb is nonzero.}
\item{lat:}{Initial north/south or 'y' coordinate of agent. Should start with -16.}
\item{long:}{Initial east/west of 'x' coordinate of agent. Should start with 145.}
\item{clutchSize:}{Number of juveniles in the clutch.}
\item{enzyme:}{Enzyme Kinetic Score of agents. Initialised uniform randomly. FIX}
\item{pDeath:}{Probability of death.}
} @param Njuv The number of initial juvenile agents. @return A
data.table of Njuv juvenile agents.

\begin{Shaded}
\begin{Highlighting}[]
\NormalTok{initialise_juveniles <-}\StringTok{ }\ControlFlowTok{function}\NormalTok{(Njuv)\{}
\NormalTok{  juv.dt <-}\StringTok{ }\KeywordTok{setNames}\NormalTok{(}\KeywordTok{data.frame}\NormalTok{(}\KeywordTok{matrix}\NormalTok{(}\DataTypeTok{ncol =} \DecValTok{8}\NormalTok{, }\DataTypeTok{nrow =}\NormalTok{ Njuv)),}
          \KeywordTok{c}\NormalTok{(}\StringTok{"motherID"}\NormalTok{,}\StringTok{"fatherID"}\NormalTok{,}\StringTok{"age"}\NormalTok{,}\StringTok{"stage"}\NormalTok{ ,}\StringTok{"infProb"}\NormalTok{, }\StringTok{"clutchSize"}\NormalTok{,}\StringTok{"enzyme"}\NormalTok{,}\StringTok{"pDeath"}\NormalTok{))}
\NormalTok{  juv.dt}\OperatorTok{$}\NormalTok{motherID <-}\StringTok{ }\DecValTok{-1}
\NormalTok{  juv.dt}\OperatorTok{$}\NormalTok{fatherID <-}\StringTok{ }\DecValTok{-1}
\NormalTok{  juv.dt}\OperatorTok{$}\NormalTok{age      <-}\StringTok{ }\KeywordTok{as.integer}\NormalTok{(}\KeywordTok{lapply}\NormalTok{(juv.dt}\OperatorTok{$}\NormalTok{age, }\ControlFlowTok{function}\NormalTok{(x) x <-}\StringTok{ }\KeywordTok{round}\NormalTok{(}\KeywordTok{runif}\NormalTok{(}\DecValTok{1}\NormalTok{, }\DataTypeTok{min=}\DecValTok{0}\NormalTok{, }\DataTypeTok{max=}\DecValTok{14}\NormalTok{),}\DecValTok{0}\NormalTok{))) }\CommentTok{#Uniform between 0 and 14 as a quick fix- FIX}
\NormalTok{  juv.dt}\OperatorTok{$}\NormalTok{stage    <-}\StringTok{ }\KeywordTok{mapply}\NormalTok{(}\DataTypeTok{FUN =}\NormalTok{ init_juv_stage, juv.dt}\OperatorTok{$}\NormalTok{age)}

  \CommentTok{#this can be replaced with a binary for inf/noninf since fringe cases lead to CI}
\NormalTok{  juv.dt}\OperatorTok{$}\NormalTok{infProb  <-}\StringTok{ }\DecValTok{0} \CommentTok{#dummy setup, this will be sampled in RABC (also need correct value from Carla)}

\NormalTok{  juv.dt}\OperatorTok{$}\NormalTok{clutchSize <-}\StringTok{ }\NormalTok{param}\OperatorTok{$}\NormalTok{eta_}\DecValTok{1} \CommentTok{#since these will always be wild type}
\NormalTok{  juv.dt}\OperatorTok{$}\NormalTok{enzyme   <-}\StringTok{ }\KeywordTok{as.double}\NormalTok{(}\KeywordTok{lapply}\NormalTok{(juv.dt}\OperatorTok{$}\NormalTok{enzyme, }\ControlFlowTok{function}\NormalTok{(x) x <-}\StringTok{ }\KeywordTok{runif}\NormalTok{(}\DecValTok{1}\NormalTok{, }\DataTypeTok{min=}\DecValTok{0}\NormalTok{, }\DataTypeTok{max=}\FloatTok{0.95}\NormalTok{))) }\CommentTok{#CHANGE: this should not be uniform}
\NormalTok{  juv.dt}\OperatorTok{$}\NormalTok{pDeath   <-}\StringTok{ }\NormalTok{param}\OperatorTok{$}\NormalTok{alpha_j }\CommentTok{#fix this. see removeNatDeath in Carla's code}

  \CommentTok{##BUT for the moment we're just using the same method as the adult lat/long to get working prototype}
  \CommentTok{#juv.dt$lat  <- lapply(juv.dt$lat, function (x) x<-round(runif(1, min = min(boundaryDat$Lat), max = max(boundaryDat$Lat)),6))}
  \CommentTok{#juv.dt$long <- lapply(juv.dt$long, function (x) x<-round(runif(1, min = min(boundaryDat$Long), max = max(boundaryDat$Long)),6))}

  \CommentTok{#The following two lines calculate lat, long and gridID}
\NormalTok{  position.dt <-}\StringTok{ }\KeywordTok{init_position}\NormalTok{(boundaryDat, Njuv)}
\NormalTok{  juv.dt      <-}\StringTok{ }\KeywordTok{cbind}\NormalTok{(juv.dt,position.dt)}


  \KeywordTok{return}\NormalTok{(juv.dt)}
\NormalTok{\}}
\end{Highlighting}
\end{Shaded}

Creates a data.table of newly laid eggs to add to the juvenile
data.table.

Input is the indices of mothers attempting to lay eggs. Much of this
code is regarding the handling of CI: \itemize{
 \item{Neither mother nor father carry Wolbachia: no Wolbachia in offspring}
 \item{Mother carries Wolbachia: offspring have nonzero probability of carrying}
 \item{Mother does not carry, father does: offspring suffer CI and won't hatch}
} Latitude/Longitude of agents is determined as the same position that
the mother is currently at. Clutch sizes, as per literature, differ
depending on Wolbachia status of mother. This is handled by the ABC
parameter eta\_1 and eta\_2.

@section data.table variables and initialisation: \describe{
\item{motherID:}   {ID of mother.}
\item{fatherID:}   {ID of father.}
\item{age:}        {Age in days. Initialised at 0 since they're new!}
\item{stage:}      {Development stage of clutch. Should only be 1 since these are eggs.}
\item{infProb:}    {Probability of carrying Wolbachia. 0: no Wolbachia, -1: Cytoplasmic
                  Incompatability, else infProb is nonzero.}
\item{lat:}        {Initial north/south or 'y' coordinate of agent.
                  Should start with -16. Should be same as mother.}
\item{long:}       {Initial east/west of 'x' coordinate of agent.
                  Should start with 145. Should be same as mother.}
\item{clutchSize:} {Number of juveniles in the clutch.}
\item{pDeath:}     {Probability of death.}
} @param toLay List of the indices of mothers attempting to lay a
clutch. @param eta\_1 Clutch size of non-Wolbachia carrying mothers.
@param eta\_2 Clutch size Wolbachia carrying mothers. @param p\_1
Probability of complete maternal Wolbachia transmission. @param alpha\_j
Juvenile mortality rate. @return A data.table of juvenile agents in
stage 1 corresponding to each mother in toLay.

\begin{Shaded}
\begin{Highlighting}[]
\NormalTok{initialise_eggs <-}\StringTok{ }\ControlFlowTok{function}\NormalTok{(toLay, eta_}\DecValTok{1}\NormalTok{, eta_}\DecValTok{2}\NormalTok{, p_}\DecValTok{1}\NormalTok{, alpha_j)\{}
\NormalTok{  noMothers <-}\StringTok{ }\KeywordTok{length}\NormalTok{(toLay)}
  \CommentTok{#' Subset mozzie.dt to just include mothers who are laying}
  \CommentTok{#' this makes it easy to assign variable values on to children}
\NormalTok{  mothers.dt <-}\StringTok{ }\KeywordTok{filter}\NormalTok{(mozzie.dt, ID }\OperatorTok\StringTok{ }\NormalTok{toLay)}
\NormalTok{  eggs.dt <-}\StringTok{ }\KeywordTok{data.table}\NormalTok{(}\DataTypeTok{motherID =} \KeywordTok{numeric}\NormalTok{(noMothers),}
                        \DataTypeTok{fatherID =} \KeywordTok{numeric}\NormalTok{(noMothers),}
                        \DataTypeTok{motherStatus =} \KeywordTok{numeric}\NormalTok{(noMothers),}
                        \DataTypeTok{fatherStatus =} \KeywordTok{numeric}\NormalTok{(noMothers),}
                        \DataTypeTok{age =} \KeywordTok{numeric}\NormalTok{(noMothers),}
                        \DataTypeTok{stage =} \KeywordTok{numeric}\NormalTok{(noMothers),}
                        \DataTypeTok{infProb =} \KeywordTok{numeric}\NormalTok{(noMothers) ,}
                        \DataTypeTok{lat =} \KeywordTok{double}\NormalTok{(noMothers),}
                        \DataTypeTok{long =} \KeywordTok{double}\NormalTok{(noMothers),}
                        \DataTypeTok{clutchSize =}  \KeywordTok{numeric}\NormalTok{(noMothers),}
                        \DataTypeTok{enzyme =} \KeywordTok{double}\NormalTok{(noMothers),}
                        \DataTypeTok{pDeath =} \KeywordTok{double}\NormalTok{(noMothers),}
                        \DataTypeTok{gridID =} \KeywordTok{integer}\NormalTok{(noMothers))}


  \CommentTok{# 28/6/20: changing this to use MOTHER ID and NOT THE INDEX YOU GET FROM which()}

  \CommentTok{#eggs.dt$motherID <- toLay}
  \CommentTok{#eggs.dt$fatherID <- mozzie.dt$mateID[toLay]}

  \CommentTok{#'try and use mothers.dt as much as possible as this ensures we}
  \CommentTok{#'are keeping everything in the right order}
\NormalTok{  eggs.dt}\OperatorTok{$}\NormalTok{motherID <-}\StringTok{ }\NormalTok{mothers.dt}\OperatorTok{$}\NormalTok{ID}
\NormalTok{  eggs.dt}\OperatorTok{$}\NormalTok{fatherID <-}\StringTok{ }\NormalTok{mothers.dt}\OperatorTok{$}\NormalTok{mateID}
  \CommentTok{#eggs.dt$fatherID <- mozzie.dt$mateID[mozzie.dt$ID  %in% toLay]}

\NormalTok{  eggs.dt}\OperatorTok{$}\NormalTok{age    <-}\StringTok{ }\DecValTok{0} \CommentTok{#because they're new!}
\NormalTok{  eggs.dt}\OperatorTok{$}\NormalTok{stage  <-}\StringTok{ }\DecValTok{1} \CommentTok{#because they're eggs!}
\NormalTok{  eggs.dt}\OperatorTok{$}\NormalTok{enzyme <-}\StringTok{ }\DecValTok{0} \CommentTok{#Since they are new eggs, they haven't had the "chance" to accumulate enzyme yet}
  \CommentTok{#eggs.dt$lat    <- mozzie.dt$lat[toLay] #mother's position}
  \CommentTok{#eggs.dt$long   <- mozzie.dt$long[toLay] #mother's position}

  \CommentTok{# eggs.dt$lat  <- mozzie.dt$lat[mozzie.dt$ID  %in% toLay]}
  \CommentTok{# eggs.dt$long <- mozzie.dt$long[mozzie.dt$ID  %in% toLay]}

  \CommentTok{#' Clutch will be initialised at the same position as its mother}
\NormalTok{  eggs.dt}\OperatorTok{$}\NormalTok{lat    <-}\StringTok{ }\NormalTok{mothers.dt}\OperatorTok{$}\NormalTok{lat}
\NormalTok{  eggs.dt}\OperatorTok{$}\NormalTok{long   <-}\StringTok{ }\NormalTok{mothers.dt}\OperatorTok{$}\NormalTok{long}
\NormalTok{  eggs.dt}\OperatorTok{$}\NormalTok{gridID <-}\StringTok{ }\NormalTok{mothers.dt}\OperatorTok{$}\NormalTok{gridID}

  \CommentTok{# motherStatus is just the mother's infStatus}
\NormalTok{  eggs.dt}\OperatorTok{$}\NormalTok{motherStatus <-}\StringTok{ }\NormalTok{mothers.dt}\OperatorTok{$}\NormalTok{infStatus}

  \CommentTok{#clutch size determined by mother's infStatus:}
\NormalTok{  eggs.dt}\OperatorTok{$}\NormalTok{clutchSize[}\KeywordTok{which}\NormalTok{(eggs.dt}\OperatorTok{$}\NormalTok{motherStatus }\OperatorTok{==}\StringTok{ }\DecValTok{1}\NormalTok{)] <-}\StringTok{ }\NormalTok{eta_}\DecValTok{2}
\NormalTok{  eggs.dt}\OperatorTok{$}\NormalTok{clutchSize[}\KeywordTok{which}\NormalTok{(eggs.dt}\OperatorTok{$}\NormalTok{motherStatus }\OperatorTok{==}\StringTok{ }\DecValTok{0}\NormalTok{)] <-}\StringTok{ }\NormalTok{eta_}\DecValTok{1}

  \CommentTok{#eggs.dt$gridID <- as.integer(mapply(FUN = get_gridID, eggs.dt$lat, eggs.dt$long))}

  \CommentTok{# This code needs to be removed- fix code so I don't have to do this --------}
  \CommentTok{# 21-09-20: I actually don't think we need this anymore: I think I fixed it :)}
\NormalTok{  eggs.dt[eggs.dt}\OperatorTok{==}\StringTok{"NULL"}\NormalTok{] <-}\StringTok{ }\DecValTok{0}
\NormalTok{  eggs.dt <-}\StringTok{ }\KeywordTok{na.omit}\NormalTok{(eggs.dt) }\CommentTok{#CHANGE THIS it shouldnt be happening}
  \CommentTok{#eggs.dt <- eggs.dt[complete.cases(eggs.dt$infProb),]}

  \CommentTok{#eggs.dt$clutchSize[which(mozzie.dt$infStatus[toLay] == 1)] <- eta_2 #finish}
  \CommentTok{#eggs.dt$clutchSize[which(mozzie.dt$infStatus[toLay] == 0)] <- eta_1 #finish}
\CommentTok{#  eggs.dt$clutchSize[which(mozzie.dt$infStatus[mozzie.dt$ID  %in% toLay] == 1)] <- eta_2 #finish}
\CommentTok{#  eggs.dt$clutchSize[which(mozzie.dt$infStatus[mozzie.dt$ID  %in% toLay] == 0)] <- eta_1 #finish}


  \CommentTok{# CI ----}
       \CommentTok{#eggs.dt$fatherStatus <- mozzie.dt$infStatus[which(mozzie.dt$ID %in% mozzie.dt$mateID[toLay])] #should give a list of 1s and 0s corresponding to their father's Wolbachia status}

       \CommentTok{#eggs.dt$fatherStatus <- mozzie.dt$infStatus[mozzie.dt$mateID[toLay]]}
       \CommentTok{#eggs.dt$motherStatus <- mozzie.dt$infStatus[toLay] #same but for mother}

  \CommentTok{# 5/7/20 THIS WORKS.... for one number}
  \CommentTok{#eggs.dt$fatherStatus[1] <- mozzie.dt$infStatus[which(mozzie.dt$ID %in% eggs.dt$fatherID[1])]}
  \CommentTok{# SO honestly.... just do a loop for now}

  \CommentTok{# 21-09-20 replace this:}
  \CommentTok{# eggs.dt$motherStatus <- mozzie.dt$infStatus[mozzie.dt$ID %in% toLay] # mother's Wolbachia status}
  \CommentTok{#}
  \CommentTok{# f.IDs       <- eggs.dt$fatherID # List of father IDs}
  \CommentTok{# alive.f.IDs <- which(f.IDs %in% mozzie.dt$ID) # Which eggs have alive fathers}
  \CommentTok{# grave.f.IDs <- which(f.IDs %in% graveyard$ID) # Which eggs have fathers in the graveyard}
  \CommentTok{# end replacement for 21-09-20}

  \CommentTok{#'below is me adapting the above for optimisation}
  \CommentTok{#'The idea here is to split eggs.dt into those who have alive dads and thos who}
  \CommentTok{#'have dead dads. I'll then get the wolbachia status (carry/not carry) for each agent}
  \CommentTok{#'and then rowbind them together at the end.}
\NormalTok{  alive.dads <-}\StringTok{ }\KeywordTok{filter}\NormalTok{(eggs.dt, fatherID }\OperatorTok\StringTok{ }\NormalTok{mozzie.dt}\OperatorTok{$}\NormalTok{ID)}
\NormalTok{  dead.dads <-}\StringTok{ }\KeywordTok{filter}\NormalTok{(eggs.dt, fatherID }\OperatorTok\StringTok{ }\NormalTok{graveyard}\OperatorTok{$}\NormalTok{ID)}
  \CommentTok{#'----- end adaptation}

  \CommentTok{# We need to do two loops: one for those who have alive fathers and one for those whose fathers are in the}
\CommentTok{#}
\CommentTok{#   system.time(}
\CommentTok{#   for(i in alive.f.IDs)\{}
\CommentTok{#     eggs.dt$fatherStatus[i] <- mozzie.dt$infStatus[which(mozzie.dt$ID %in% eggs.dt$fatherID[i])]}
\CommentTok{#   \})}
\CommentTok{#}
\CommentTok{#   # Now for those with dads in the graveyard}
\CommentTok{#   system.time(}
\CommentTok{#   for(j in grave.f.IDs)\{}
\CommentTok{#     eggs.dt$fatherStatus[j] <- graveyard$infStatus[which(graveyard$ID %in% eggs.dt$fatherID[j])]}
\CommentTok{#   \})}


  \CommentTok{#' EDIT 21-9-20: VECTORISE THE ABOVE CODE}
  \CommentTok{#' We need to do two loops: one for those who have alive fathers and one for those whose fathers are dead}
  \CommentTok{#' r is a vector of the father's infStatus for every juvenile clutch in alive.dads.}
 \KeywordTok{system.time}\NormalTok{(}
\NormalTok{  r <-}\StringTok{ }\KeywordTok{foreach}\NormalTok{ (}\DataTypeTok{i =} \DecValTok{1}\OperatorTok{:}\KeywordTok{length}\NormalTok{(alive.dads}\OperatorTok{$}\NormalTok{fatherID), }\DataTypeTok{.combine =}\NormalTok{ c) }\OperatorTok\StringTok{ }\NormalTok{\{}
\NormalTok{    mozzie.dt}\OperatorTok{$}\NormalTok{infStatus[}\KeywordTok{which}\NormalTok{(mozzie.dt}\OperatorTok{$}\NormalTok{ID }\OperatorTok{==}\StringTok{ }\NormalTok{alive.dads}\OperatorTok{$}\NormalTok{fatherID[i])]}
\NormalTok{  \})}
\NormalTok{  alive.dads}\OperatorTok{$}\NormalTok{fatherStatus <-}\StringTok{ }\NormalTok{r}
  \CommentTok{#' Now for those with dads in the graveyard}

 \KeywordTok{system.time}\NormalTok{(}
\NormalTok{  s <-}\StringTok{ }\KeywordTok{foreach}\NormalTok{ (}\DataTypeTok{j =} \DecValTok{1}\OperatorTok{:}\KeywordTok{length}\NormalTok{(dead.dads}\OperatorTok{$}\NormalTok{fatherID), }\DataTypeTok{.combine =}\NormalTok{ c) }\OperatorTok\StringTok{ }\NormalTok{\{}
\NormalTok{    graveyard}\OperatorTok{$}\NormalTok{infStatus[}\KeywordTok{which}\NormalTok{(graveyard}\OperatorTok{$}\NormalTok{ID }\OperatorTok{==}\StringTok{ }\NormalTok{dead.dads}\OperatorTok{$}\NormalTok{fatherID[j])]}
\NormalTok{  \})}
\NormalTok{  dead.dads}\OperatorTok{$}\NormalTok{fatherStatus <-}\StringTok{ }\NormalTok{s}

  \CommentTok{# smoosh these back together}
\NormalTok{  l <-}\StringTok{ }\KeywordTok{list}\NormalTok{(alive.dads, dead.dads)}
\NormalTok{  eggs.dt <-}\StringTok{ }\KeywordTok{rbindlist}\NormalTok{(l, }\DataTypeTok{use.names =} \OtherTok{TRUE}\NormalTok{)}
  \CommentTok{#eggs.dt <- rbind(alive.dads, dead.dads)}


  \CommentTok{# -----}

              \CommentTok{# testing out code 5/7/20: remove later ----}
              \CommentTok{#test <- eggs.dt$fatherID}
              \CommentTok{#test.l <- which(test %in% mozzie.dt$ID) # which eggs have alive fathers}
              \CommentTok{#test.g <- which(test %in% graveyard$ID) # which eggs have fathers in the graveyard}

              \CommentTok{#test.lf <- eggs.dt$fatherID[test.l] #fatherIDs of alive fathers}
              \CommentTok{#test.gf <- eggs.dt$fatherID[test.g] #fatherIDs of dead fathers}

              \CommentTok{#which mozzies have the ID of those alive fathers}
              \CommentTok{#fatherstat.lf <- mozzie.dt$infStatus[test.lf %in% mozzie.dt$fatherID]}
              \CommentTok{# ok so we need to handle dads that are in the graveyard....}
              \CommentTok{#fatherstat.l <- mozzie.dt$infStatus[test.l]}
              \CommentTok{#fatherstat.g <- graveyard$infStatus[test.g]}

              \CommentTok{#eggs.dt$fatherStatus[test.l] <- mozzie.dt$infStatus[test %in% mozzie.dt$ID]}

              \CommentTok{#eggs.dt$fatherStatus <- mozzie.dt$infStatus[which(mozzie.dt$ID %in% eggs.dt$fatherID)] #THIS AINT WORKIN}
              \CommentTok{# End 5/7/20 playing around ----}

  \CommentTok{# Case I: If mother is a Wolbachia carrier then offspring should be (with probability p_1)}
  \CommentTok{#eggs.dt$infProb[which(eggs.dt$motherStatus == 1)] <- (1-p_1)}
\NormalTok{  eggs.dt}\OperatorTok{$}\NormalTok{infProb[}\KeywordTok{which}\NormalTok{(eggs.dt}\OperatorTok{$}\NormalTok{motherStatus }\OperatorTok{==}\StringTok{ }\DecValTok{1}\NormalTok{)] <-}\StringTok{ }\NormalTok{p_}\DecValTok{1}
\NormalTok{  eggs.dt}\OperatorTok{$}\NormalTok{pDeath[}\KeywordTok{which}\NormalTok{(eggs.dt}\OperatorTok{$}\NormalTok{motherStatus }\OperatorTok{==}\StringTok{ }\DecValTok{1}\NormalTok{)]  <-}\StringTok{ }\NormalTok{alpha_j }\CommentTok{#natural death rate}

  \CommentTok{#FIX : HACKY FOR LOOP for ANZIAM 2/2/19}

\NormalTok{  motheruninf <-}\StringTok{ }\KeywordTok{which}\NormalTok{(eggs.dt}\OperatorTok{$}\NormalTok{motherStatus }\OperatorTok{==}\StringTok{ }\DecValTok{0}\NormalTok{)}


  \CommentTok{# This is to remove the confusing if/then statements below}
\NormalTok{  which.wt <-}\StringTok{ }\KeywordTok{which}\NormalTok{(eggs.dt}\OperatorTok{$}\NormalTok{motherStatus }\OperatorTok{==}\StringTok{ }\DecValTok{0} \OperatorTok{&}\StringTok{ }\NormalTok{eggs.dt}\OperatorTok{$}\NormalTok{fatherStatus }\OperatorTok{==}\StringTok{ }\DecValTok{0}\NormalTok{) }\CommentTok{#Which are Wild Type/no Wolbachia}
\NormalTok{  eggs.dt}\OperatorTok{$}\NormalTok{infProb[which.wt] <-}\StringTok{ }\DecValTok{0}
\NormalTok{  eggs.dt}\OperatorTok{$}\NormalTok{pDeath[which.wt] <-}\StringTok{ }\NormalTok{alpha_j}


\NormalTok{  which.ci <-}\StringTok{ }\KeywordTok{which}\NormalTok{(eggs.dt}\OperatorTok{$}\NormalTok{motherStatus }\OperatorTok{==}\StringTok{ }\DecValTok{0} \OperatorTok{&}\StringTok{ }\NormalTok{eggs.dt}\OperatorTok{$}\NormalTok{fatherStatus }\OperatorTok{==}\StringTok{ }\DecValTok{1}\NormalTok{)}

\NormalTok{  eggs.dt}\OperatorTok{$}\NormalTok{infProb[which.ci] <-}\StringTok{ }\DecValTok{-1}

\NormalTok{  eggs.dt}\OperatorTok{$}\NormalTok{pDeath[which.ci] <-}\StringTok{ }\DecValTok{-1}

 \CommentTok{# print("get up to here")}
  \CommentTok{#below is deprecated- remove}
\CommentTok{#  for(i in 1:length(motheruninf))\{}
\CommentTok{#    if(eggs.dt$fatherStatus[motheruninf[i]] == 0)\{}
\CommentTok{#      eggs.dt$infProb[i] <- 0}
\CommentTok{#      eggs.dt$pDeath <- alpha_j}
\CommentTok{#    \}}
\CommentTok{#    else if(eggs.dt$fatherStatus[motheruninf[i]] == 1)\{}
\CommentTok{#      eggs.dt$infProb[i] <- -1}
\CommentTok{#      eggs.dt$pDeath[i] <- -1}
\CommentTok{#    \}}
\CommentTok{#    else\{}
      \CommentTok{#error handling case; offspring are just wild CHECK}
\CommentTok{#      eggs.dt$infProb[i] <- 0}
\CommentTok{#      eggs.dt$pDeath <- alpha_j}
\CommentTok{#    \}}
\CommentTok{#  \}}
  \CommentTok{#Get rid of "extra information" in data table that we no longer need}
\NormalTok{  to.drop <-}\StringTok{ }\KeywordTok{c}\NormalTok{(}\StringTok{"motherStatus"}\NormalTok{,}\StringTok{"fatherStatus"}\NormalTok{)}
  \CommentTok{#eggs.dt <- eggs.dt[,c("motherStatus","fatherStatus") := NULL] #this only works if new.eggs.dt is purely a data.table.}
  \CommentTok{#it seems like some kind of data.frame and data.table chimera, I'm terrified but will ignore it}
\NormalTok{  eggs.dt <-}\StringTok{ }\NormalTok{eggs.dt[ , }\OperatorTok{!}\NormalTok{(}\KeywordTok{names}\NormalTok{(eggs.dt) }\OperatorTok\StringTok{ }\NormalTok{to.drop)]}

  \KeywordTok{return}\NormalTok{(eggs.dt)}
\NormalTok{\}}
\end{Highlighting}
\end{Shaded}

Simulates a release of Wolbachia-carrying mosquitoes. Releases in this
experiment are always adults. Much of this is functionally similar to
initialise\_adults. Since this is a release, some data is made up/dummy:
for example, we don't know the ID of the parents. Latitude/Longitude of
agents is given by data supplied by the WMP.

@section Data.table variables and initialisation: \describe{
\item{ID:}{Unique ID number of agent.}
\item{gender:}{Male is 0, female is 1. We know proportions of female/males released
         from data.}
\item{mateID:}{Unique ID of their mate. Since no initial mosquitoes will have a mate
        yet, it is initialised as -1.
        Males will always have mateID as -1 since they can have multiple mates.}
\item{enzyme:}{Enzyme Kinetic Score. See init for explanation. Initialised
        uniform randomly.}
\item{age:}{Age in days. Initialised uniform randomly around what we would expect
        young adults to be.}
\item{gonoCycle:}{Gonotrophic cycle. Means something different for males and females.
           males: number of times they've mated in a day, to be reset daily
           females: how many times they've laid a clutch of eggs
           we roughly estimate the gonoCycle of females based on age.}
\item{timeDeath:}{Timestep they died: initialised as -1 as they are alive.}
\item{typeDeath:}{Random mortality/trapped death/death due to old age: which type?}
\item{whereTrapped:}{In the event of trapped death, where did they die? -1 otherwise.}
\item{motherID:}{Unique ID of mother. -1 since wild release.}
\item{fatherID:}{Unique ID of father. -1 since wild release.}
\item{infStatus:}{1 if they carry Wolbachia, 0 if no Wolbachia, -1 for CI.
           They should mostly be 1 with a few 0 due to incomplete transmission.}
\item{lat:}{Initial north/south or 'y' coordinate of agent. Should start with -16.}
\item{long:}{Initial east/west of 'x' coordinate of agent. Should start with 145.}
} @param noReleased The number of initial adult agents. @param noMale
Number of males. @param noFemale Number of females. @param lat Latitude
of release site. @param long Longitude of release site. @param idStart
Index starting for new mosquito ID numbers. @param GID Unique identifier
for a release site. From data. @return A data.table of N adult agents.

\begin{Shaded}
\begin{Highlighting}[]
\NormalTok{initialise_release <-}\StringTok{ }\ControlFlowTok{function}\NormalTok{(noReleased, noMale, noFemale, lat, long, idStart, GID)\{}
\NormalTok{  propInfRelease <-}\StringTok{ }\DecValTok{1} \CommentTok{#proportion of released mosquitoes that carry Wolbachia (CHECK: get actual value from WMP)}
\NormalTok{  noVariables <-}\StringTok{ }\DecValTok{16} \CommentTok{#number of variables used to track state of mozzie minus 2 (lat & long). this is so I make sure to remember to change it as necessary}
  \CommentTok{#18/7: change release.dt so we just have a numeric age for mosquitoes rather than timeBirth/timeAdult/timeDeath}
\NormalTok{  release.dt <-}\StringTok{ }\KeywordTok{setNames}\NormalTok{(}\KeywordTok{data.frame}\NormalTok{(}\KeywordTok{matrix}\NormalTok{(}\DataTypeTok{ncol =}\NormalTok{ noVariables, }\DataTypeTok{nrow =}\NormalTok{ noReleased)), }\KeywordTok{c}\NormalTok{(}\StringTok{"ID"}\NormalTok{, }\StringTok{"gender"}\NormalTok{, }\StringTok{"lat"}\NormalTok{,}\StringTok{"long"}\NormalTok{,}\StringTok{"mateID"}\NormalTok{, }\StringTok{"enzyme"}\NormalTok{,}\StringTok{"age"}\NormalTok{,}\StringTok{"gonoCycle"}\NormalTok{,}\StringTok{"timeDeath"}\NormalTok{,}\StringTok{"typeDeath"}\NormalTok{,}\StringTok{"whereTrapped"}\NormalTok{,}\StringTok{"motherID"}\NormalTok{,}\StringTok{"fatherID"}\NormalTok{,}\StringTok{"infStatus"}\NormalTok{,}\StringTok{"releaseLoc"}\NormalTok{,}\StringTok{"gridID"}\NormalTok{))}
\CommentTok{#  release.dt <- data.table(ID = integer(noReleased), gender = numeric(noReleased), mateID = numeric(noReleased), enzyme = numeric(noReleased), age = numeric(noReleased), gonoCycle = numeric(noReleased), timeDeath = numeric(noReleased), typeDeath = numeric(noReleased), whereTrapped = numeric(noReleased), motherID = numeric(noReleased), fatherID = numeric(noReleased), infStatus = numeric(noReleased), releaseLoc = numeric(noReleased), long = numeric(noReleased), lat = numeric(noReleased), gridID = integer(noReleased) )}


  \CommentTok{#print(release.dt)}
\NormalTok{  release.dt}\OperatorTok{$}\NormalTok{gender <-}\StringTok{ }\DecValTok{0} \CommentTok{#Initialise all as male}
\NormalTok{  release.dt}\OperatorTok{$}\NormalTok{gender[}\DecValTok{1}\OperatorTok{:}\NormalTok{noFemale] <-}\StringTok{ }\DecValTok{1} \CommentTok{#Make #noFemale entries female}
  \CommentTok{#release.dt$gender <- lapply(release.dt$gender, function(x) x <- rbinom(1,1,1 - pmale)) #probability of male is calculated above. since female mozzies are represented by 1 (a success) we have 1-pmale}
  \CommentTok{#release.dt$ID     <- (max(mozzie.dt$ID) + 1):((max(mozzie.dt$ID)) + noReleased) #unique ID for each mosquito #CHECK}
\NormalTok{  release.dt}\OperatorTok{$}\NormalTok{ID <-}\StringTok{ }\NormalTok{idStart}\OperatorTok{:}\NormalTok{(idStart}\OperatorTok{+}\NormalTok{noReleased}\DecValTok{-1}\NormalTok{) }\CommentTok{#Unique ID for each mosquito}

  \CommentTok{#release.dt$age <- lapply(release.dt$age, function (x) x<-round(rtruncnorm(1,mean=20,sd=2,a=14,b=30)))}
\NormalTok{  release.dt}\OperatorTok{$}\NormalTok{age        <-}\StringTok{ }\KeywordTok{as.integer}\NormalTok{(}\KeywordTok{lapply}\NormalTok{(release.dt}\OperatorTok{$}\NormalTok{age, }\ControlFlowTok{function}\NormalTok{(x) x <-}\StringTok{ }\KeywordTok{round}\NormalTok{(}\KeywordTok{runif}\NormalTok{(}\DecValTok{1}\NormalTok{,}\DataTypeTok{min=}\DecValTok{18}\NormalTok{,}\DataTypeTok{max=}\DecValTok{21}\NormalTok{),}\DecValTok{0}\NormalTok{)))}
\NormalTok{  release.dt}\OperatorTok{$}\NormalTok{motherID   <-}\StringTok{ }\DecValTok{-1}
\NormalTok{  release.dt}\OperatorTok{$}\NormalTok{fatherID   <-}\StringTok{ }\DecValTok{-1}
\NormalTok{  release.dt}\OperatorTok{$}\NormalTok{releaseLoc <-}\StringTok{ }\NormalTok{GID}
  \CommentTok{#release.dt$long <- 145.760}
  \CommentTok{#release.dt$lat <- -16.918}
\NormalTok{  release.dt}\OperatorTok{$}\NormalTok{mateID     <-}\StringTok{ }\DecValTok{-1}
\NormalTok{  release.dt}\OperatorTok{$}\NormalTok{long       <-}\StringTok{ }\NormalTok{long}
\NormalTok{  release.dt}\OperatorTok{$}\NormalTok{lat        <-}\StringTok{ }\NormalTok{lat}
\NormalTok{  releaseGrid           <-}\StringTok{ }\KeywordTok{get_gridID}\NormalTok{(}\DataTypeTok{testlat =}\NormalTok{ lat, }\DataTypeTok{testlong =}\NormalTok{ long) }\CommentTok{#Every mozzie released will be at the same grid}
\NormalTok{  release.dt}\OperatorTok{$}\NormalTok{gridID     <-}\StringTok{ }\NormalTok{releaseGrid}
\NormalTok{  release.dt}\OperatorTok{$}\NormalTok{timeDeath  <-}\StringTok{ }\DecValTok{-1}
\NormalTok{  release.dt}\OperatorTok{$}\NormalTok{typeDeath  <-}\StringTok{ }\DecValTok{-1}
  \CommentTok{#release.dt$infStatus    <- lapply(release.dt$infStatus, function(x) x <-rbinom(1,1,propInfRelease))}
\NormalTok{  release.dt}\OperatorTok{$}\NormalTok{infStatus  <-}\StringTok{ }\DecValTok{1}
  \CommentTok{#infStatus: 1 for wolbachia, 0 for no wolbachia, -1 for CI}
\NormalTok{  release.dt}\OperatorTok{$}\NormalTok{whereTrapped <-}\StringTok{ }\DecValTok{-1}

\NormalTok{  release.dt}\OperatorTok{$}\NormalTok{enzyme <-}\StringTok{ }\KeywordTok{as.double}\NormalTok{(}\KeywordTok{lapply}\NormalTok{(release.dt}\OperatorTok{$}\NormalTok{enzyme, }\ControlFlowTok{function}\NormalTok{(x) x <-}\StringTok{ }\KeywordTok{runif}\NormalTok{(}\DecValTok{1}\NormalTok{, }\DataTypeTok{min=}\DecValTok{0}\NormalTok{, }\DataTypeTok{max=}\DecValTok{1}\NormalTok{)))  }\CommentTok{#CHANGE: want to use initial age distribution to get spread of initial enzymes}

\NormalTok{  release.dt}\OperatorTok{$}\NormalTok{gonoCycle[}\KeywordTok{which}\NormalTok{(release.dt}\OperatorTok{$}\NormalTok{gender }\OperatorTok{==}\StringTok{ }\DecValTok{0}\NormalTok{)] <-}\StringTok{ }\DecValTok{0} \CommentTok{#males start off at 0 because 'gonoCycle' tracks number of mating events in a day for males}
\NormalTok{  release.dt}\OperatorTok{$}\NormalTok{gonoCycle[}\KeywordTok{which}\NormalTok{(release.dt}\OperatorTok{$}\NormalTok{gender }\OperatorTok{==}\StringTok{ }\DecValTok{1} \OperatorTok{&}\StringTok{ }\NormalTok{release.dt}\OperatorTok{$}\NormalTok{age }\OperatorTok{>=}\DecValTok{14} \OperatorTok{&}\StringTok{ }\NormalTok{release.dt}\OperatorTok{$}\NormalTok{age }\OperatorTok{<}\StringTok{ }\DecValTok{20}\NormalTok{)] <-}\StringTok{ }\DecValTok{0}
\NormalTok{  release.dt}\OperatorTok{$}\NormalTok{gonoCycle[}\KeywordTok{which}\NormalTok{(release.dt}\OperatorTok{$}\NormalTok{gender }\OperatorTok{==}\StringTok{ }\DecValTok{1} \OperatorTok{&}\StringTok{ }\NormalTok{release.dt}\OperatorTok{$}\NormalTok{age }\OperatorTok{>=}\DecValTok{20} \OperatorTok{&}\StringTok{ }\NormalTok{release.dt}\OperatorTok{$}\NormalTok{age }\OperatorTok{<}\StringTok{ }\DecValTok{26}\NormalTok{)] <-}\StringTok{ }\DecValTok{1}
\NormalTok{  release.dt}\OperatorTok{$}\NormalTok{gonoCycle[}\KeywordTok{which}\NormalTok{(release.dt}\OperatorTok{$}\NormalTok{gender }\OperatorTok{==}\StringTok{ }\DecValTok{1} \OperatorTok{&}\StringTok{ }\NormalTok{release.dt}\OperatorTok{$}\NormalTok{age }\OperatorTok{>=}\DecValTok{26}\NormalTok{)] <-}\StringTok{ }\DecValTok{2}

  \KeywordTok{return}\NormalTok{(release.dt)}

\NormalTok{\}}
\end{Highlighting}
\end{Shaded}

Initialises graveyard data.table The graveyard is where the entries of
all dead adult mozzies go @return An empty data.table that's ready to be
rbind()-ed

\begin{Shaded}
\begin{Highlighting}[]
\NormalTok{initialise_graveyard <-}\StringTok{ }\ControlFlowTok{function}\NormalTok{()\{}

\NormalTok{  graveyard <-}\StringTok{ }\KeywordTok{data.table}\NormalTok{(}
      \DataTypeTok{ID           =} \KeywordTok{integer}\NormalTok{(),}
      \DataTypeTok{gender       =} \KeywordTok{integer}\NormalTok{(),}
      \DataTypeTok{mateID       =} \KeywordTok{double}\NormalTok{(),}
      \DataTypeTok{enzyme       =} \KeywordTok{double}\NormalTok{(),}
      \DataTypeTok{age          =} \KeywordTok{integer}\NormalTok{(),}
      \DataTypeTok{gonoCycle    =} \KeywordTok{numeric}\NormalTok{(),}
      \DataTypeTok{timeDeath    =} \KeywordTok{numeric}\NormalTok{(),}
      \DataTypeTok{typeDeath    =} \KeywordTok{numeric}\NormalTok{(),}
      \DataTypeTok{whereTrapped =} \KeywordTok{double}\NormalTok{(),}
      \DataTypeTok{motherID     =} \KeywordTok{numeric}\NormalTok{(),}
      \DataTypeTok{fatherID     =} \KeywordTok{numeric}\NormalTok{(),}
      \DataTypeTok{infStatus    =} \KeywordTok{numeric}\NormalTok{(),}
      \DataTypeTok{releaseLoc   =} \KeywordTok{numeric}\NormalTok{(),}
      \DataTypeTok{long         =} \KeywordTok{double}\NormalTok{(),}
      \DataTypeTok{lat          =} \KeywordTok{double}\NormalTok{(),}
      \DataTypeTok{gridID       =} \KeywordTok{numeric}\NormalTok{()}
\NormalTok{  )}
\KeywordTok{return}\NormalTok{(graveyard)}
\NormalTok{\}}
\end{Highlighting}
\end{Shaded}

Initialises juvenile graveyard data.table The entries here are for egg
clutches that won't hatch due to CI This does not include juveniles that
died of natural causes (since that's just a number in the juv.dt
data.table) @return Initialised data.table of the juvenile graveyard.

\begin{Shaded}
\begin{Highlighting}[]
\NormalTok{initialise_juv_graveyard <-}\StringTok{ }\ControlFlowTok{function}\NormalTok{()\{}

\NormalTok{ juv.graveyard <-}\StringTok{ }\KeywordTok{data.table}\NormalTok{(}
   \DataTypeTok{motherID   =} \KeywordTok{numeric}\NormalTok{(),}
   \DataTypeTok{fatherID   =} \KeywordTok{numeric}\NormalTok{(),}
   \DataTypeTok{age        =} \KeywordTok{integer}\NormalTok{(),}
   \DataTypeTok{stage      =} \KeywordTok{numeric}\NormalTok{(),}
   \DataTypeTok{infProb    =} \KeywordTok{numeric}\NormalTok{(),}
   \DataTypeTok{clutchSize =} \KeywordTok{numeric}\NormalTok{(),}
   \DataTypeTok{enzyme     =} \KeywordTok{double}\NormalTok{(),}
   \DataTypeTok{pDeath     =} \KeywordTok{numeric}\NormalTok{(),}
   \DataTypeTok{long       =} \KeywordTok{double}\NormalTok{(),}
   \DataTypeTok{lat        =} \KeywordTok{double}\NormalTok{(),}
   \DataTypeTok{gridID     =} \KeywordTok{integer}\NormalTok{()}
\NormalTok{   )}

  \KeywordTok{return}\NormalTok{(juv.graveyard)}
\NormalTok{\}}
\end{Highlighting}
\end{Shaded}

\end{document}
